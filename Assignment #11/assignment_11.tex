\documentclass[12pt]{article}
 
\usepackage[margin=1in]{geometry}
\usepackage{amsmath,amsthm,amssymb, mathtools}
\usepackage[T1]{fontenc}
\usepackage{lmodern}
\usepackage{fixltx2e}
\usepackage[shortlabels]{enumitem}
\usepackage{mathrsfs}
 
\newcommand{\N}{\mathbb{N}}
\newcommand{\R}{\mathbb{R}}
\newcommand{\Z}{\mathbb{Z}}
\newcommand{\Q}{\mathbb{Q}}
 
\newenvironment{theorem}[2][Theorem]{\begin{trivlist}
\item[\hskip \labelsep {\bfseries #1}\hskip \labelsep {\bfseries #2.}]}{\end{trivlist}}
\newenvironment{lemma}[2][Lemma]{\begin{trivlist}
\item[\hskip \labelsep {\bfseries #1}\hskip \labelsep {\bfseries #2.}]}{\end{trivlist}}
\newenvironment{exercise}[2][Exercise]{\begin{trivlist}
\item[\hskip \labelsep {\bfseries #1}\hskip \labelsep {\bfseries #2.}]}{\end{trivlist}}
\newenvironment{problem}[2][Problem]{\begin{trivlist}
\item[\hskip \labelsep {\bfseries #1}\hskip \labelsep {\bfseries #2.}]}{\end{trivlist}}
\newenvironment{question}[2][Question]{\begin{trivlist}
\item[\hskip \labelsep {\bfseries #1}\hskip \labelsep {\bfseries #2.}]}{\end{trivlist}}
\newenvironment{corollary}[2][Corollary]{\begin{trivlist}
\item[\hskip \labelsep {\bfseries #1}\hskip \labelsep {\bfseries #2.}]}{\end{trivlist}}
\newcommand{\textfrac}[2]{\dfrac{\text{#1}}{\text{#2}}}

\begin{document}

\title{Real Analysis I: Assignment 11}

\author{Chris Hayduk}
\date{December 5, 2019}

\maketitle

\begin{problem}{1}
\end{problem}

Suppose $f$ is measurable, non-negative, and $\int f = 0$.. Let $E = \{x: f(x) > 0\}$. Thus, if we define $E_n = \{x: f(x) \geq \frac{1}{n}\}$, we have that $\cup E_n = E$.\\

We know from Proposition 5 that $\int_{E_n} f = 0 \geq \left(\frac{1}{n}\right) mE_n$ for every $n$.\\

Since measure is non-negative and $\frac{1}{n}$ is always positive, $0 \geq \left(\frac{1}{n}\right) mE_n \implies mE_n = 0$ for every $n$.\\

As a result, we have that $mE = 0$. That is, the set of all $x$ where $f(x) = 0$ has measure 0. Hence, $f = 0$ almost everywhere.

\begin{problem}{2}
\end{problem}

\begin{problem}{3}
\end{problem}

\begin{problem}{4}
\end{problem}

Let $f_n = \chi_{[n, n+1]}$. It is clear that for each $n$, $f_n$ equals 1 on $[n, n+1]$ and equals 0 everywhere else.\\

Since for any given $x$, $f_n(x) = 1$ for at most two values of $n$ (if $x = n$ for some endpoint $n$), we have that
\begin{align*}
\lim_{n \to \infty} f_n(x) = 0 \; \forall x
\end{align*} 

Thus, we have that
\begin{align*}
\int \lim f_n = \int 0 = 0
\end{align*}

Now fix $n \in \mathbb{N}$. Observe that
\begin{align*}
\int f_n = 1 \times m([n, n+1]) = 1 (n + 1 - n) = 1
\end{align*}

Since this holds for every $n \in \mathbb{N}$, we have that $\underline{\lim} \int f_n = 1$. Thus, we can see that,
\begin{align*}
\int \lim f_n = 0 < \underline{\lim} \int f_n = 1
\end{align*}

\end{document}