\documentclass[12pt]{article}
 
\usepackage[margin=1in]{geometry}
\usepackage{amsmath,amsthm,amssymb, mathtools}
\usepackage[T1]{fontenc}
\usepackage{lmodern}
\usepackage{fixltx2e}
\usepackage[shortlabels]{enumitem}
\usepackage{mathrsfs}
 
\newcommand{\N}{\mathbb{N}}
\newcommand{\R}{\mathbb{R}}
\newcommand{\Z}{\mathbb{Z}}
\newcommand{\Q}{\mathbb{Q}}
 
\newenvironment{theorem}[2][Theorem]{\begin{trivlist}
\item[\hskip \labelsep {\bfseries #1}\hskip \labelsep {\bfseries #2.}]}{\end{trivlist}}
\newenvironment{lemma}[2][Lemma]{\begin{trivlist}
\item[\hskip \labelsep {\bfseries #1}\hskip \labelsep {\bfseries #2.}]}{\end{trivlist}}
\newenvironment{exercise}[2][Exercise]{\begin{trivlist}
\item[\hskip \labelsep {\bfseries #1}\hskip \labelsep {\bfseries #2.}]}{\end{trivlist}}
\newenvironment{problem}[2][Problem]{\begin{trivlist}
\item[\hskip \labelsep {\bfseries #1}\hskip \labelsep {\bfseries #2.}]}{\end{trivlist}}
\newenvironment{question}[2][Question]{\begin{trivlist}
\item[\hskip \labelsep {\bfseries #1}\hskip \labelsep {\bfseries #2.}]}{\end{trivlist}}
\newenvironment{corollary}[2][Corollary]{\begin{trivlist}
\item[\hskip \labelsep {\bfseries #1}\hskip \labelsep {\bfseries #2.}]}{\end{trivlist}}
\newcommand{\textfrac}[2]{\dfrac{\text{#1}}{\text{#2}}}

\begin{document}

\title{Real Analysis I: Assignment 4}

\author{Chris Hayduk}
\date{October 3, 2019}

\maketitle

\begin{problem}{1}
\end{problem}

Let $<F_n>$ be a sequence of non-empty, closed sets of real numbers with the property that $F_{n+1} \subset F_{n}$ for all $n \in \mathbb{N}$. Furthermore, assume there exists an $m$ such that $F_m$ is bounded.\\

Since $F_m$ is bounded we know that $\exists c \in \mathbb{R}$ such that $\forall x \in F_m, |x| \leq c$. Now examine $F_{m+1}$. We know that $F_{m+1} \neq \emptyset$ and $F_{m+1} \subset F_{m}$. That is, $x \in F_{m+1} \implies x \in F_{m}$. As a result, every $x \in F_{m+1}$ is bounded by the same $c$.\\

This same line of reasoning applies for any $n \geq m$ since we have that $F_m \supset F_{m+1} \supset \cdots$ and $F_n \neq \emptyset \; \forall n$. Thus, we now have a subsequence of the original sequence which consists solely of closed, bounded, nonempty subsets.\\

We know that $\forall k \geq m, F_k$ is a non-empty, closed, and bounded set. As a result each $F_k$ must contain its infimum, which we can denote by $x_k$.\\

We can see that $x_k \leq c$ and $x_k \leq x_{k+1}$. Thus, $<x_k>$ is a monotonically increasing, bounded sequence. By the monotone convergence theorem, this sequence converges to a limit point $x$.\\

Since the sets are nested, for any k, $x_j \in F_k \; \forall j \geq k$. Furthermore, $F_k$ is closed, so we have $x \in F_k$. This applies for any $k \geq m$, so $x \in \bigcap^\infty_{i=m} F_i$.\\

Now that we have considered the subsequence of non-empty, closed, and bounded sets let us incorporate the finite collection of unbounded sets $F_1, \cdots, F_{m-1}$.\\

We know that $F_m \subset F_{m-1} \subset \cdots F_{1}$. Thus, $F_{m} \subset F_n \; \forall n < m$.\\

As a result, we clearly have $\bigcap^{m-1}_{i=0} F_i = F_m$.\\

Now if we combine the two intersections, we have
\begin{align*}
&\bigcap^\infty_{i=0} F_i = \bigcap^{m-1}_{j=0} F_j \; \cap \; \bigcap^\infty_{k=m} F_k\\
\implies &(F_1 \cap F_2 \cap \cdots \cap F_{m-1}) \cap (F_{m} \cap F_{m+1} \cap \cdots)\\
\implies &F_{m} \cap (F_{m} \cap F_{m+1} \cap \cdots)\\
\implies &(F_m \cap F_m) \cap F_{m+1} \cap \cdots\\
\implies &F_m \cap F_{m+1} \cap \cdots = \bigcap^\infty_{k=m} F_k
\end{align*}

Since we know that $x \in \bigcap^\infty_{i=m} F_i$, the derivation above implies that $x \in \bigcap^\infty_{i=0} F_i \neq \emptyset$ as well. Thus, the intersection of a nested sequence of non-empty, closed sets of real numbers is non-empty if one of the sets is bounded.\\

Now let $F_n = \{x \in \mathbb{R}: x \geq n\} = [n, \infty)$. Clearly each $F_n \subset \mathbb{R}$ is unbounded and closed.\\ 

Furthermore, we have that $F_{n+1} = \{x \in \mathbb{R}: x \geq n+1\} = [n+1, \infty)$. Then we can see that $F_{n+1} \subset F_n$ since $[n+1, \infty) \subset [n, \infty)$.\\

Now assume $\cap F_n \neq \emptyset$ and let $x \in \cap F_n$. Then $x \in F_n \; \forall n$. Since $F_n \subset \mathbb{R}$, we have that $x \in \mathbb{R}$. Thus, by the Axiom of Archimedes, $\exists k \in \mathbb{N}$ such that $k > x$.\\

Hence, by the definition of $F_n$, we have that $x \not\in F_k$. However, this contradicts the assumption that $x \in \cap F_n$ and, by extension, that $\cap F_n \neq \emptyset$.\\

As a result, if there is not at least one bounded $F_n$, we have that $\cap F_n = \emptyset$.

\begin{problem}{2}
\end{problem}

Let $A = \{x \in [a, b]: f(x) \leq \gamma\}$. We have that $a \in A$ and $A$ is bounded above by $b$. As a result, the supremum of $A$ exists. Let $c = \sup A$.\\

We claim that $f(c) = \gamma$.\\

We know that $f$ is continuous, so choose some $\epsilon > 0$. Then $\exists \delta > 0$ such that $|f(x) - f(c)| < \epsilon$ whenever $|x - c| < \delta$.\\

This yields,
\begin{align*}
f(x) - \epsilon < f(c) < f(x) + \epsilon
\end{align*}

Since $c$ is the supremum of the set, there exists $c' \in (c - \delta, c]$ such that
\begin{align*}
&f(c) + \epsilon < f(c')\\
\implies &f(c) < f(c') + \epsilon \leq \gamma + \epsilon
\end{align*}

In addition, there exists $c'' \in (c, c + \delta)$ such that 
\begin{align*}
&f(c'') < f(c) + \epsilon\\
\implies &f(c) > f(c'') - \epsilon \geq \gamma - \epsilon
\end{align*}

Combining these two inequalities yields, 
\begin{align*}
\gamma - \epsilon \leq f(c) \leq \gamma + \epsilon
\end{align*}

Since $\epsilon > 0$ can be arbitrarily small, $f(c) = \gamma$.

\begin{problem}{3}
\end{problem}

Firstly, let $g(x) = f(x) \; \forall x \in F$.\\

Since $F$ is closed, we know that $\tilde{F}$ is open. By Proposition 8, $\tilde{F}$ is the union of a countable collection of disjoint open intervals. Define $g$ on $\tilde{F}$ to be linear on these open intervals. Then $g$ is defined on all of $\mathbb{R}$.\\

We already know that $g$ is continuous on $F$, so it suffices to show that $g$ is continuous on $\tilde{F}$.\\

Let $(x_1, x_2)$ be one of the open intervals whose union equals $\tilde{F}$. Since $g$ is linear, we have that $x \in (x_1, x_2) \implies g(x) = ax + b$ for some constants $a, b \in \mathbb{R}$. Assume $a \neq 0$.\\

Let $\epsilon = \delta|a|$. Then there is a $\delta > 0$ such that for $x, y \in (x_1, x_2)$,
\begin{align*}
|x - y| < \delta &\implies |g(x) - g(y)| < \epsilon\\
&\implies |ax + b - (ay + b)| < \epsilon\\
&\implies |a(x - y)| < \epsilon\\
&\implies |x - y| < \frac{\epsilon}{|a|} = \frac{\delta|a|}{|a|} = \delta
\end{align*}

Thus, $g(x)$ is continuous on $(x_1, x_2)$ when $a \neq 0$.\\

Moreover, if $a = 0$, then $g(x) = b \; \forall x \in (x_1, x_2)$. Thus, $|g(x) - g(y)| = 0 < \epsilon \; \forall x, y \in (x_1, x_2)$ and $\forall \epsilon > 0$. Hence, $g(x)$ is continuous on $(x_1, x_2)$ for any choice of $a, b \in \mathbb{R}$.\\

Since $g(x)$ is continuous for any choice of $a, b \in \mathbb{R}$, we can choose $a, b$ such that $g(x_1) = f(x_1)$ and $g(x_2) = f(x_2)$. (We know that $x_1, x_2 \in F$ by the disjoint property of the open intervals). This yields $a = \frac{f(x_1) - f(x_2)}{x_1 - x_2}$ and $b = f(x_1) - x_1\frac{f(x_1) - f(x_2)}{x_1 - x_2}$\\

By this construction, $g(x)$ is continuous on $[x_1, x_2]$ and $g(x_1) = f(x_1), g(x_2) = f(x_2)$.\\

Since $(x_1, x_2)$ is an arbitrary open interval from the collection of countable open intervals whose union equals $\tilde{F}$, we have that $g(x)$ is continuous on all such intervals with the property that $g(x_1) = f(x_1), g(x_2) = f(x_2)$ for the endpoints of said intervals.\\

We know that $g(x)$ is continuous at all interior points of $F$ and $\tilde{F}$. Thus, we must show that $g(x)$ is continuous at the boundary points of $F$, which are precisely the boundary points of $\tilde{F}$. Namely, the endpoints of each open interval.\\

Choose a boundary point $x$ and an $\epsilon > 0$. Since $g(x)$ is continuous in $F$ and $x \in F$, there exists a $\delta_1 > 0$ such that $|x - y| < \delta \implies |g(x) - g(y)| < \epsilon \; \forall y \in F$. Furthermore, by the construction above of $g(x)$ on $\tilde{F}$, there exists a $\delta_2 > 0$ such that $|x - y| < \delta_2 \implies |g(x) - g(y)| < \epsilon \; \forall y \in \tilde{F}$. Take $\delta = \min\{\delta_1, \delta_2\}$, and we can clearly see that $g(x)$ is continuous at $x$. Since $x$ is an arbitrary boundary point of $F$, $g(x)$ is continuous at all such boundary points (ie. the endpoints of each open interval comprising $\tilde{F}$).\\

Hence, we have that $g(x)$ is continuous at all points in $F$ and $\tilde{F}$. Thus, $g$ is continuous on $F \cup \tilde{F} = \mathbb{R}$.

\begin{problem}{4}
\end{problem}

Since $f$ is continuous on $[a, b]$ (a closed and bounded subset of $\mathbb{R}$), we have that $f$ is uniformly continuous.\\

Let $\epsilon > 0$. Then there exists a $\delta > 0$ such that $|f(x) - f(y)| < \frac{\epsilon}{2}$ for all $x, y \in [a, b]$.\\

Choose $N \in \mathbb{N}$ such that $\frac{b-a}{N} < \delta$ and let $x_i = a + i\frac{b-a}{N}$. Furthermore, construct $\varphi$ similarly to how we constructed $g$ in the above problem: let $\varphi$ be linear on each $[x_i, x_{i+1}]$ with the property that $\varphi(x_i) = f(x_i) \; \forall i$.\\

Now let $x \in [a, b]$. Then $\exists i$ such that $x \in [x_i, x_{i+1}]$. In addition, assume $f(x_i) \leq f(x_{i+1})$, which implies that $\varphi(x_i) \leq \varphi(x) \leq \varphi(x_{i+1})$ by properties of linear functions. Then,
\begin{align*}
|\varphi(x) - f(x)| &\leq |\varphi(x) - \varphi(x_i)| + |\varphi(x_i) - f(x)|\\
&\leq |\varphi(x_{i+1}) - \varphi(x_i)| + |f(x_i) - f(x)| < \frac{\epsilon}{2} + \frac{\epsilon}{2} = \epsilon
\end{align*}

If $f(x_i) > f(x_{i+1})$, we have that $\varphi(x_i) > \varphi(x) > \varphi(x_{i+1})$, yielding,
\begin{align*}
|\varphi(x) - f(x)| &< |\varphi(x) - \varphi(x_{i+1})| + |\varphi(x_{i+1}) - f(x)|\\
&< |\varphi(x_i) - \varphi(x_{i+1})| + |f(x_{i+1}) - f(x)| < \frac{\epsilon}{2} + \frac{\epsilon}{2} = \epsilon
\end{align*}

Thus, the inequality holds regardless of the inequality relationship between endpoints.\\

Since $x \in [a, b]$ was arbitrary, we have that $\varphi: [a,b] \to \mathbb{R}$ is a polygonal function with the property that $|f(x) - \varphi(x)| < \epsilon \; \forall x \in [a, b]$.

\end{document}