\documentclass[12pt]{article}
 
\usepackage[margin=1in]{geometry}
\usepackage{amsmath,amsthm,amssymb, mathtools}
\usepackage[T1]{fontenc}
\usepackage{lmodern}
\usepackage{fixltx2e}
\usepackage[shortlabels]{enumitem}
\usepackage{mathrsfs}
 
\newcommand{\N}{\mathbb{N}}
\newcommand{\R}{\mathbb{R}}
\newcommand{\Z}{\mathbb{Z}}
\newcommand{\Q}{\mathbb{Q}}
 
\newenvironment{theorem}[2][Theorem]{\begin{trivlist}
\item[\hskip \labelsep {\bfseries #1}\hskip \labelsep {\bfseries #2.}]}{\end{trivlist}}
\newenvironment{lemma}[2][Lemma]{\begin{trivlist}
\item[\hskip \labelsep {\bfseries #1}\hskip \labelsep {\bfseries #2.}]}{\end{trivlist}}
\newenvironment{exercise}[2][Exercise]{\begin{trivlist}
\item[\hskip \labelsep {\bfseries #1}\hskip \labelsep {\bfseries #2.}]}{\end{trivlist}}
\newenvironment{problem}[2][Problem]{\begin{trivlist}
\item[\hskip \labelsep {\bfseries #1}\hskip \labelsep {\bfseries #2.}]}{\end{trivlist}}
\newenvironment{question}[2][Question]{\begin{trivlist}
\item[\hskip \labelsep {\bfseries #1}\hskip \labelsep {\bfseries #2.}]}{\end{trivlist}}
\newenvironment{corollary}[2][Corollary]{\begin{trivlist}
\item[\hskip \labelsep {\bfseries #1}\hskip \labelsep {\bfseries #2.}]}{\end{trivlist}}
\newcommand{\textfrac}[2]{\dfrac{\text{#1}}{\text{#2}}}

\begin{document}

\title{Real Analysis I: Assignment 12}

\author{Chris Hayduk}
\date{December 12, 2019}

\maketitle

\begin{problem}{1}
\end{problem}

\begin{itemize}

\item  $D^+f(0)$:
\begin{align*}
D^+f(0) &= \overline{\lim}_{h \to 0^+} \frac{f(0+h) - f(0)}{h}\\
&= \overline{\lim}_{h \to 0^+} \frac{f(h) - f(0)}{h}\\
&= \overline{\lim}_{h \to 0^+} \frac{f(h) - 0}{h}\\
&= \overline{\lim}_{h \to 0^+} \frac{h\sin(\frac{1}{h})}{h}\\
&= \overline{\lim}_{h \to 0^+} \sin\left(\frac{1}{h}\right)
\end{align*}

Since $\sin\left(\frac{1}{x}\right)$ oscillates between -1 and 1 infinitely often as we approach 0, we have that,
\begin{align*}
D^+f(0) = \overline{\lim}_{h \to 0^+} \sin\left(\frac{1}{h}\right) = 1
\end{align*}

\item $D^-f(0)$:
\begin{align*}
D^-f(0) &= \overline{\lim}_{h \to 0^+} \frac{f(0) - f(0-h)}{h}\\
&= \overline{\lim}_{h \to 0^+} \frac{f(0) - f(-h)}{h}\\
&= \overline{\lim}_{h \to 0^+} \frac{0 - f(-h)}{h}\\
&= \overline{\lim}_{h \to 0^+} \frac{-(-h)\sin(\frac{1}{-h})}{h}\\
&= \overline{\lim}_{h \to 0^+} \sin\left(\frac{1}{-h}\right)
\end{align*}

Since we know $\sin\left(\frac{1}{x}\right)$ oscillates between -1 and 1 infinitely often as we approach 0, we have that $\sin\left(\frac{1}{-x}\right)$ oscillates between -1 and 1 infinitely as well. Hence,
\begin{align*}
D^-f(0) = \overline{\lim}_{h \to 0^+} \sin\left(\frac{1}{-h}\right) = 1
\end{align*}

By the same reasoning from the two above,

\item $D_+f(0)$:

\begin{align*}
D_+f(0) &= \underline{\lim}_{h \to 0^+} \frac{f(0+h) - f(0)}{h}\\
&= \underline{\lim}_{h \to 0^+} \frac{f(h) - f(0)}{h}\\
&= \underline{\lim}_{h \to 0^+} \frac{f(h) - 0}{h}\\
&= \underline{\lim}_{h \to 0^+} \frac{h\sin(\frac{1}{h})}{h}\\
&= \underline{\lim}_{h \to 0^+} \sin\left(\frac{1}{h}\right)\\
&= -1
\end{align*}

\item $D_-f(0)$:

\begin{align*}
D_-f(0) &= \underline{\lim}_{h \to 0^+} \frac{f(0) - f(0-h)}{h}\\
&= \underline{\lim}_{h \to 0^+} \frac{f(0) - f(-h)}{h}\\
&= \underline{\lim}_{h \to 0^+} \frac{0 - f(-h)}{h}\\
&= \underline{\lim}_{h \to 0^+} \frac{-(-h)\sin(\frac{1}{-h})}{h}\\
&= \underline{\lim}_{h \to 0^+} \sin\left(\frac{1}{-h}\right)\\
&= -1
\end{align*}
\end{itemize}

\newpage
\begin{problem}{2}
\end{problem}

Since we know that $f$ assumes a local minimum at $c$, we know that $f(c) \leq f(c+h)$ for $h \in \mathbb{R}$ sufficiently small. As a result, we have that $f(c) - f(c-h) \leq 0$ and $f(c+h) - f(c) \geq 0$.\\

This result implies that,
\begin{align*}
\frac{f(c+h) - f(c)}{h} \geq 0
\end{align*}

and

\begin{align*}
\frac{f(c) - f(c-h)}{h} \leq 0
\end{align*}

for $h > 0$ and $h$ sufficiently close to $0$. Thus, if we take limits with $h$ approaching $0$ from above, these inequalities are preserved. Hence,
\begin{align*}
D_-f(c) &= \underline{\lim}_{h \to 0^+} \frac{f(c) - f(c-h)}{h} \leq 0\\
D^-f(c) &= \overline{\lim}_{h \to 0^+} \frac{f(c) - f(c-h)}{h} \leq 0\\
D_+f(c) &= \underline{\lim}_{h \to 0^+} \frac{f(c+h) - f(c)}{h} \geq 0\\
D^+f(c) &= \overline{\lim}_{h \to 0^+} \frac{f(c+h) - f(c)}{h} \geq 0
\end{align*}

Now, by the definitions of $\liminf$ and $\limsup$, we also have that
\begin{align*}
D_-f(c) \leq D^-f(c)
\end{align*}

and
\begin{align*}
D_+f(c) \leq D^+f(c)
\end{align*}

Putting it all together, we have
\begin{align*}
D_-f(c) \leq D^-f(c) \leq 0 \leq D_+f(c) \leq D^+f(c)
\end{align*}

when $f$ attains a local minimum at $c$.\\

If $f$ attains a local minimum at $a$, we know from the above that $0 \leq D_+f(a) \leq D^+f(a)$ since $\frac{f(a+h) - f(a)}{h} \geq 0$ for small, positive values of $h$. However, we cannot make a statement about $D^-f(a)$ or $D_-f(a)$ because we do not know how the function behaves for values of $x < a$.\\

Similarly, if $f$ attains a local minimum at $b$, we know from the above that $D_-f(b) \leq D^-f(b) \leq 0$ since $\frac{f(b) - f(b-h)}{h} \leq 0$ for small, positive values of $h$. However, we cannot make a statement about $D^+f(b)$ or $D_+f(b)$ because we do not know how the function behaves for values of $x > b$.

\newpage
\begin{problem}{3}
\end{problem}

Let $g$ be a function on $[a, b]$ with $D^+g(x) \geq \epsilon$ for some $\epsilon > 0$ and for all values of $x \in (a, b)$. Then we have,
\begin{align*}
D^+g(x) = \overline{\lim}_{h \to 0^+} \frac{g(x+h) - g(x)}{h} \geq \epsilon
\end{align*}

Suppose that $g$ is not non-decreasing. That is, there exist $x, y \in [a, b]$ with $x < y$ such that $g(x) > g(y)$.\\

By similar reasoning to question 2, we know that $D^+f(c) \leq 0$ when some function $f$ attains a maximum at some point $c$. Since $D^+g(x) \geq \epsilon > 0$ for all $x \in (a, b)$, $g$ does not attain a local maximum on $(a, b)$. Hence, $g$ is decreasing on $(a, y]$ and $D^+g(c) \leq 0$ for all $c \in (a, y)$, a contradiction. Thus, $g$ must be non-decreasing on $[a, b]$.\\

Now let $f(x)$ be a continuous function on $[a, b]$ and assume $D^+f \geq 0$ on $(a, b)$.\\

For any $\epsilon > 0$, $D^+(f(x) + \epsilon x) \geq \epsilon$ on $(a, b)$. Thus, $f(x) + \epsilon x$ is non-decreasing on $[a, b]$.\\

Now let $x < y$. $f$ non-decreasing implies that $f(x) + \epsilon x \leq f(y) + \epsilon y$.\\

Suppose for contradiction that $f(x) > f(y)$. Then, by the above inequality, we have
\begin{align*}
0 < f(x) - f(y) \leq \epsilon(x - y)
\end{align*}

Now choose $\epsilon = (f(x) - f(y))/(2(y-x))$. This yields,
\begin{align*}
f(x) - f(y) \leq (f(x) - f(y))/2
\end{align*}

This is a contradiction. Thus, when $x < y$, we have that $f(x) \leq f(y)$ and so $f$ is non-decreasing on $[a, b]$.

\begin{problem}{4}
\end{problem}

For any $x$, we have that,
\begin{align*}
D^+(f+g)(x) &= \overline{\lim}_{h \to 0^+} \frac{(f+g)(x+h) - (f+g)(x)}{h}\\
&= \overline{\lim}_{h \to 0^+} \frac{f(x+h) + g(x+h) - f(x) - g(x)}{h}\\
&= \overline{\lim}_{h \to 0^+} \frac{f(x+h) - f(x) + g(x+h) - g(x)}{h}\\
&= \overline{\lim}_{h \to 0^+} \left[\frac{f(x+h) - f(x)}{h} + \frac{g(x+h) - g(x)}{h}\right]\\
&\leq \overline{\lim}_{h \to 0^+} \frac{f(x+h) - f(x)}{h} + \overline{\lim}_{h \to 0^+} \frac{g(x+h) - g(x)}{h}\\
&= D^+f + D^+g
\end{align*}

\end{document}