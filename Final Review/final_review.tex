\documentclass[12pt]{article}
 
\usepackage[margin=1in]{geometry}
\usepackage{amsmath,amsthm,amssymb, mathtools}
\usepackage[T1]{fontenc}
\usepackage{lmodern}
\usepackage{fixltx2e}
\usepackage[shortlabels]{enumitem}
\usepackage{mathrsfs}
 
\newcommand{\N}{\mathbb{N}}
\newcommand{\R}{\mathbb{R}}
\newcommand{\Z}{\mathbb{Z}}
\newcommand{\Q}{\mathbb{Q}}
 
\newenvironment{theorem}[2][Theorem]{\begin{trivlist}
\item[\hskip \labelsep {\bfseries #1}\hskip \labelsep {\bfseries #2.}]}{\end{trivlist}}
\newenvironment{lemma}[2][Lemma]{\begin{trivlist}
\item[\hskip \labelsep {\bfseries #1}\hskip \labelsep {\bfseries #2.}]}{\end{trivlist}}
\newenvironment{exercise}[2][Exercise]{\begin{trivlist}
\item[\hskip \labelsep {\bfseries #1}\hskip \labelsep {\bfseries #2.}]}{\end{trivlist}}
\newenvironment{problem}[2][Problem]{\begin{trivlist}
\item[\hskip \labelsep {\bfseries #1}\hskip \labelsep {\bfseries #2.}]}{\end{trivlist}}
\newenvironment{question}[2][Question]{\begin{trivlist}
\item[\hskip \labelsep {\bfseries #1}\hskip \labelsep {\bfseries #2.}]}{\end{trivlist}}
\newenvironment{corollary}[2][Corollary]{\begin{trivlist}
\item[\hskip \labelsep {\bfseries #1}\hskip \labelsep {\bfseries #2.}]}{\end{trivlist}}
\newcommand{\textfrac}[2]{\dfrac{\text{#1}}{\text{#2}}}

\begin{document}

\title{Real Analysis I: Final Exam Review}

\author{Chris Hayduk}
\date{December 17, 2019}

\maketitle

\section{Definitions}

\begin{itemize}

\item $\sigma\text{-algebra}$:\\

A collection $\mathscr{A}$ of subsets of $X$ is called an $\textbf{algebra}$ of sets or a $\textbf{Boolean algebra}$ if (i) $A \cup B$ is in $\mathscr{A}$ whenever $A$ and $B$ are, and (ii) $\tilde{A}$ is in $\mathscr{A}$ whenever A is.\\

$\mathscr{A}$ is called a $\mathbf{\sigma}\textbf{-algebra}$, or a $\textbf{Borel field}$, if it has the above properties $\textit{and}$ every union of a countable collection of sets in $\mathscr{A}$ is again in $\mathscr{A}$
\\
\item Uniform convergence of a sequence of functions:\\

A sequence $\langle f_n \rangle$ of functions defined on a set $E$ is said to converge $\textbf{uniformly}$ on E if given $\epsilon > 0$, there is an $N$ such that for all $x \in E$ and all $n \geq N$, we have $|f(x) - f_n(x)| < \epsilon$.
\\
\item Borel sets:\\

The collection $\mathscr{B}$ of Borel sets is the smallest $\sigma$-algebra which contains all of the open sets.
\\
\item $F_{\sigma}$ set:\\

An $F_{\sigma}$ set is a countable union of closed sets.
\\
\item $G_{\delta}$ set:\\

A $G_{\delta}$ set is a countable intersection of open sets.
\\
\item Outer measure:\\

$m^*A = \inf_{A \subset \cup I_n} \Sigma \ell(I_n)$, where $\{I_n\}$ represents a countable collections of open intervals that cover $A$.
\\
\item Measurable set:\\

A set $E$ is said to be $\textbf{measurable}$ if for each set $A$ we have $m^*A = m^*(A \cap E) + m^*(A \cap \tilde{E})$
\\
\item Measurable function:\\

Let $f$ be an extended real-valued function whose domain is measurable. Then the following statements are equivalent:
\begin{enumerate}
\item For each real number $\alpha$, the set $\{x: f(x) > \alpha\}$ is measurable.
\item For each real number $\alpha$, the set $\{x: f(x) \geq \alpha\}$ is measurable.
\item For each real number $\alpha$, the set $\{x: f(x) < \alpha\}$ is measurable.
\item For each real number $\alpha$, the set $\{x: f(x) \leq \alpha\}$ is measurable.
\end{enumerate}
These statements imply that, for each extended real number $\alpha$, the set $\{x: f(x) = \alpha\}$ is measurable.

An extended real-valued function $f$ is said to be (Lebesgue) measurable if its domain is measurable and if it satisfies one of the first four statements above.
\\
\item Almost everywhere:\\

A property is said to hold $\textbf{almost everywhere}$ (abbreviated a.e.) if the set of points where it fails to hold is a set of measure zero.
\\
\item Lebesgue integral of simple functions:\\

Let $\phi = \Sigma_{i=1}^n a_i \chi_{E_i}$ with $E_i \cap E_j = \emptyset$ for $i \neq j$. Suppose each set $E_i$ is a measurable set of finite measure. Then,
\begin{align*}
\int \phi = \Sigma_{i=1}^n a_i mE_i
\end{align*}
\\
\item Lebesgue integral of bounded measurable functions that vanish outside of a set of  finite measure:\\

If $f$ is a bounded measurable function defined on a measurable set $E$ with $mE$ finite, we define the (Lebesgue) integral of $f$ over $E$ by
\begin{align*}
\int_E f(x)dx = \inf \int_E \psi(x) dx
\end{align*}

for all simple functions $\phi \geq f$.
\\
\item Lebesgue integral of non-negative measurable functions:\\

If $f$ is a non-negative measurable function defined on a measurable set $E$, we define,
\begin{align*}
\int_E f = \sup_{h \leq f} \int_E h
\end{align*}

where $h$ is a bounded measurable function such that $m\{x: h(x) \neq 0\}$ is finite.
\\
\item Lebesgue integral of a general measurable function:\\

A measurable function $f$ is said to be integrable over $E$ if $f^+$ and $f^-$ are both integrable over $E$ (that is, $\int_E f^+ < \infty$ and $\int_E f^- < \infty$). In this case, we define,
\begin{align*}
\int_E f = \int_E f^+ - \int_E f^-
\end{align*}

where $f^+ = \max\{f(x), 0\}$ and $f^- = \max\{-f(x), 0\}$
\\
\item Convergence in measure:\\

A sequence $\langle f_n \rangle$ of measurable functions is said to converge to $f$ in measure if, given $\epsilon > 0$, there is an $N$ such that for all $n \geq N$ we have

\begin{align*}
m\{x: |f(x) - f_n(x)| \geq \epsilon\} < \epsilon
\end{align*}
\\
\item Vitali cover:\\

Let $g$ be a collection of intervals. Then we say that $g$ covers a set $E$ in the sense of Vitali if, for each $\epsilon > 0$ and any $x \in E$, there is an interval $I \in g$ such that $x \in I$ and $\ell(I) < \epsilon$. The intervals may be open, closed, or half-open, but we do not allow degenerate intervals consisting of only one point.
\\
\item Total variation:\\

Let $f$ be a real-valued function defined on the interval $[a, b]$, and let $a = x_0 < x_1 < \cdots < x_k = b$ be any subdivision of $[a, b]$. Define 
\begin{align*}
p &= \Sigma_{i=1}^k [f(x_i) - f(x_{i-1})]^+\\
n &= \Sigma_{i=1}^k [f(x_i) - f(x_{i-1})]^-\\
t = n + p &= \Sigma_{i=1}^k |f(x_i) - f(x_{i-1})|
\end{align*}

where we use $r^+$ to denote $r$ if $r \geq 0$ and $0$ if $r \leq 0$, and set $r^- = |r| - r^+$. We have $f(b) - f(a) = p - n$. Set,
\begin{align*}
P &= \sup p\\
N &= \sup n\\
T &= \sup t
\end{align*}

where we take the supremum over all possible subdivisions of $[a, b]$.

We clearly have $P \leq T \leq P+N$. We call P, N, T the positive, negative, and total variations of $f$ over $[a, b]$. We sometimes write $T_a^b$, $T_a^b(f)$, etc. to denote the dependence on the interval $[a, b]$ or on the function $f$.

If $T < \infty$, we say that $f$ is of bounded variation over $[a, b]$.
\end{itemize}

%%%%%%%%%%%%%%%%%%%%%%%%%%%%%%%%%%%%%%%%%%%%%%%%%%%%%%%%%%%%%%%%%%%%%%%%%%%%%%%%%%%%
\newpage
\section{Theorems}

\begin{itemize}

\item Heine-Borel Theorem:\\

Let $F$ be a closed and bounded set of real numbers. Then each open covering of $F$ has a finite subcovering.

That is, if a collection $\mathscr{C}$ is a collection of open sets such that $F \subset \cup \{O: O \in \mathscr{C}\}$, then there is a collection $\{O_1, O_2, ..., O_n\}$ of sets in $\mathscr{C}$ such that,
\begin{align*}
F \subset \cup_{i=1}^n O_i
\end{align*}
\\
\item Egoroff's Theorem:\\

If $\langle f_n \rangle$ is a sequence of measurable functions that converge to a real-valued function $f$ a.e. on a measurable set $E$ of finite measure, then given $\eta > 0$, there is a subset $A \subset E$ with $mA < \eta$ such that $f_n$ converges to $f$ uniformly on $E \sim A$
\\
\item Fatou's Lemma:\\

If $\langle f_n \rangle$ is a sequence of non-negative measurable functions and $f_n(x) \to f(x)$ almost everywhere on a set $E$, then
\begin{align*}
\int_E f \leq \underline{\lim} \int_E f_n
\end{align*}
\\
\item Monotone Convergence Theorem:\\

Let $\langle f_n \rangle$ be an increasing sequence of non-negative measurable functions, and let $f = \lim f_n$ a.e. Then,
\begin{align*}
\int f = \lim \int f_n
\end{align*}
\\
\item Lebesgue Convergence Theorem:\\

Let $g$ be integrable over $E$ and let $\langle f_n \rangle$ be a sequence of measurable functions such that $|f_n| \leq g$ on $E$ and for almost all $x$ in $E$ we have $f(x) = \lim f_n(x)$. Then,
\begin{align*}
\int_E f = \lim \int_E f_n
\end{align*}
\\
\item Vitali Covering Lemma:\\

Let $E$ be a set of finite outer measure and $g$ a collection of intervals that covers $E$ in the sense of Vitali. Then, given $\epsilon > 0$, there is a finite disjoint collection $\{I_1, ..., I_n\}$ of intervals in $g$ such that,
\begin{align*}
m^*\left[E \sim \cup_{n=1}^N I_n\right] < \epsilon
\end{align*}
\end{itemize}

%%%%%%%%%%%%%%%%%%%%%%%%%%%%%%%%%%%%%%%%%%%%%%%%%%%%%%%%%%%%%%%%%%%%%%%%%%%%%%%%%%%%
\newpage
\section{Proofs}

\begin{itemize}

\item If $f$ is measurable and $f = g$ a.e., then $g$ is also measurable:\\

Let $E$ be the set $\{x: f(x) \neq g(x)\}$. By hypothesis, $mE = 0$. Now,
\begin{align*}
\{x: g(x) > \alpha\} = \left[\{x: f(x) > \alpha\} \cup \{x \in E: g(x) > \alpha\}\right] \sim \{x \in E: g(x) \leq \alpha\}
\end{align*}

The first set on the right is measurable since $f$ is a measurable function. The last two sets on the right are measurable since they are subsets of $E$ and $mE = 0$. Thus, $\{x: g(x) > \alpha\}$ is measurable for each $\alpha$ and so $g$ is measurable.
\\
\item Bounded Convergence Theorem:

$\textbf{Theorem statement:}$ Let $\langle f_n \rangle$ be a sequence of measurable functions defined on a set $E$ of finite measure, and suppose that there is a real number $M$ such that $|f_n(x)| \leq M$ for all $n$ and all $x$. If $f(x) = \lim f_n(x)$ for each $x$ in $E$, then
\begin{align*}
\int_E f = \lim \int_E f_n
\end{align*}

$\textbf{Proof:}$

By Proposition 3.23, we have that, given $\epsilon > 0$, there is an $N$ and a measurable set $A \subset E$ with $mA < \frac{\epsilon}{4M}$ such that for $n \geq N$ and $x \in E \sim A$, we have $|f_n(x) - f(x)| < \frac{\epsilon}{2mE}$. Then,
\begin{align*}
\left|\int_E f_n - \int_E f\right| &= \left|\int_E f_n - f\right|\\
&\leq \int_E \left|f_n - f\right|\\
&= \int_{E \sim A} \left|f_n - f\right| + \int_A \left|f_n - f\right|\\
&< \frac{\epsilon}{2mE}m(E \sim A) + \frac{\epsilon}{2mE}mA\\
&< \frac{\epsilon}{2mE}mE + \frac{\epsilon}{2mE}mE\\
&= \frac{\epsilon}{2} + \frac{\epsilon}{2} = \epsilon 
\end{align*}
\\
\item Proposition 14 from Ch. 4:

$\textbf{Proposition statement:}$ Let $f$ be a non-negative function which is integrable over a set $E$. Then given $\epsilon > 0$, there is a $\delta > 0$ such that for every set $A \subset E$ with $mA < \delta$, we have
\begin{align*}
\int_A f < \epsilon
\end{align*}

$\textbf{Proof:}$

The proposition would be trivial if $f$ were bounded, so assume $f$ is unbounded. Set $f_n(x) = f(x)$ if $f(x) \leq n$ and $f_n(x) = n$ otherwise. Then each $f_n(x)$ is bounded and $f_n$ converges to $f$ at each point.

By the Monotone Convergence Theorem, there is an $N$ such that $\int_E f_N > \int_E f - \epsilon/2$ and $\int_E f - f_N < \epsilon/2$.

Choose $\delta < \frac{\epsilon}{2N}$. If $mA < \delta$, we have
\begin{align*}
\int_A f &= \int_A (f - f_N) + \int_A f_N\\
&< \int_E (f - f_N) + NmA\\
&< \frac{\epsilon}{2} + \frac{\epsilon}{2} = \epsilon
\end{align*}

as required.
\\
\item A function is of bounded variation iff it is the difference of two monotone real-valued functions:\\

Let $f$ be of bounded variation and set $g(x) = P^x_a$ and $h(x) = N_a^x$. Then $g$ and $h$ are monotone increasing functions which are real valued, since $0 \leq P_a^x \leq T_a^x \leq T_a^b < \infty$ and $0 \leq N_a^x \leq T_a^x \leq T_a^b < \infty$.

But $f(x) = g(x) - h(x) + f(a) = g(x) - [h(x) - f(a)]$ by Lemma 4. Since $h - f(a)$ is a monotone function, we have $f$ expressed as the difference of two monotone functions.

On the other hand, if $f = g - h$ on $[a, b]$ with $g$ and $h$ increasing, then for any subdivision we have
\begin{align*}
t = \Sigma |f(x_i) - f(x_{i-1})| &\leq \Sigma [g(x_i) - g(x_{i-1})] + \Sigma [h(x_i) - h(x_{i-1})]\\
&= g(b) - g(a) + h(b) - h(a) < \infty
\end{align*}

Since this holds for any subdivision of $[a, b]$, we have that $t < \infty$ for all such subdivisions. Hence, $\sup t = T < \infty$ and so $f$ is of bounded variation.
\end{itemize}

\end{document}