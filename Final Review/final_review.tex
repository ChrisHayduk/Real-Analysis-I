\documentclass[12pt]{article}
 
\usepackage[margin=1in]{geometry}
\usepackage{amsmath,amsthm,amssymb, mathtools}
\usepackage[T1]{fontenc}
\usepackage{lmodern}
\usepackage{fixltx2e}
\usepackage[shortlabels]{enumitem}
\usepackage{mathrsfs}
 
\newcommand{\N}{\mathbb{N}}
\newcommand{\R}{\mathbb{R}}
\newcommand{\Z}{\mathbb{Z}}
\newcommand{\Q}{\mathbb{Q}}
 
\newenvironment{theorem}[2][Theorem]{\begin{trivlist}
\item[\hskip \labelsep {\bfseries #1}\hskip \labelsep {\bfseries #2.}]}{\end{trivlist}}
\newenvironment{lemma}[2][Lemma]{\begin{trivlist}
\item[\hskip \labelsep {\bfseries #1}\hskip \labelsep {\bfseries #2.}]}{\end{trivlist}}
\newenvironment{exercise}[2][Exercise]{\begin{trivlist}
\item[\hskip \labelsep {\bfseries #1}\hskip \labelsep {\bfseries #2.}]}{\end{trivlist}}
\newenvironment{problem}[2][Problem]{\begin{trivlist}
\item[\hskip \labelsep {\bfseries #1}\hskip \labelsep {\bfseries #2.}]}{\end{trivlist}}
\newenvironment{question}[2][Question]{\begin{trivlist}
\item[\hskip \labelsep {\bfseries #1}\hskip \labelsep {\bfseries #2.}]}{\end{trivlist}}
\newenvironment{corollary}[2][Corollary]{\begin{trivlist}
\item[\hskip \labelsep {\bfseries #1}\hskip \labelsep {\bfseries #2.}]}{\end{trivlist}}
\newcommand{\textfrac}[2]{\dfrac{\text{#1}}{\text{#2}}}

\begin{document}

\title{Real Analysis I: Final Exam Review}

\author{Chris Hayduk}
\date{December 17, 2019}

\maketitle

\section{Definitions}

\begin{itemize}

\item $\sigma\text{-algebra}$:\\

A collection $\mathscr{A}$ of subsets of $X$ is called an $\textbf{algebra}$ of sets or a $\textbf{Boolean algebra}$ if (i) $A \cup B$ is in $\mathscr{A}$ whenever $A$ and $B$ are, and (ii) $\tilde{A}$ is in $\mathscr{A}$ whenever A is.\\

$\mathscr{A}$ is called a $\mathbf{\sigma}\textbf{-algebra}$, or a $\textbf{Borel field}$, if it has the above properties $\textit{and}$ every union of a countable collection of sets in $\mathscr{A}$ is again in $\mathscr{A}$
\\
\item Uniform convergence of a sequence of functions:\\

A sequence $\langle f_n \rangle$ of functions defined on a set $E$ is said to converge $\textbf{uniformly}$ on E if given $\epsilon > 0$, there is an $N$ such that for all $x \in E$ and all $n \geq N$, we have $|f(x) - f_n(x)| < \epsilon$.
\\
\item Borel sets:\\

The collection $\mathscr{B}$ of Borel sets is the smallest $\sigma$-algebra which contains all of the open sets.
\\
\item $F_{\sigma}$ set:\\

An $F_{\sigma}$ set is a countable union of closed sets.
\\
\item $G_{\delta}$ set:\\

A $G_{\delta}$ set is a countable intersection of open sets.
\\
\item Outer measure:\\

$m^*A = \inf_{A \subset \cup I_n} \Sigma \ell(I_n)$, where $\{I_n\}$ represents a countable collections of open intervals that cover $A$.
\\
\item Measurable set:\\

A set $E$ is said to be $\textbf{measurable}$ if for each set $A$ we have $m^*A = m^*(A \cap E) + m^*(A \cap \tilde{E})$
\\
\item Measurable function:\\

Let $f$ be an extended real-valued function whose domain is measurable. Then the following statements are equivalent:
\begin{enumerate}
\item For each real number $\alpha$, the set $\{x: f(x) > \alpha\}$ is measurable.
\item For each real number $\alpha$, the set $\{x: f(x) \geq \alpha\}$ is measurable.
\item For each real number $\alpha$, the set $\{x: f(x) < \alpha\}$ is measurable.
\item For each real number $\alpha$, the set $\{x: f(x) \leq \alpha\}$ is measurable.
\end{enumerate}
These statements imply that, for each extended real number $\alpha$, the set $\{x: f(x) = \alpha\}$ is measurable.

An extended real-valued function $f$ is said to be (Lebesgue) measurable if its domain is measurable and if it satisfies one of the first four statements above.
\\
\item Almost everywhere:\\

A property is said to hold $\textbf{almost everywhere}$ (abbreviated a.e.) if the set of points where it fails to hold is a set of measure zero.
\\
\item Lebesgue integral of simple functions:

\end{itemize}
\end{document}