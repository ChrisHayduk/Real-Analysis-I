\documentclass[12pt]{article}
 
\usepackage[margin=1in]{geometry}
\usepackage{amsmath,amsthm,amssymb, mathtools}
\usepackage[T1]{fontenc}
\usepackage{lmodern}
\usepackage{fixltx2e}
\usepackage[shortlabels]{enumitem}
\usepackage{mathrsfs}
 
\newcommand{\N}{\mathbb{N}}
\newcommand{\R}{\mathbb{R}}
\newcommand{\Z}{\mathbb{Z}}
\newcommand{\Q}{\mathbb{Q}}
 
\newenvironment{theorem}[2][Theorem]{\begin{trivlist}
\item[\hskip \labelsep {\bfseries #1}\hskip \labelsep {\bfseries #2.}]}{\end{trivlist}}
\newenvironment{lemma}[2][Lemma]{\begin{trivlist}
\item[\hskip \labelsep {\bfseries #1}\hskip \labelsep {\bfseries #2.}]}{\end{trivlist}}
\newenvironment{exercise}[2][Exercise]{\begin{trivlist}
\item[\hskip \labelsep {\bfseries #1}\hskip \labelsep {\bfseries #2.}]}{\end{trivlist}}
\newenvironment{problem}[2][Problem]{\begin{trivlist}
\item[\hskip \labelsep {\bfseries #1}\hskip \labelsep {\bfseries #2.}]}{\end{trivlist}}
\newenvironment{question}[2][Question]{\begin{trivlist}
\item[\hskip \labelsep {\bfseries #1}\hskip \labelsep {\bfseries #2.}]}{\end{trivlist}}
\newenvironment{corollary}[2][Corollary]{\begin{trivlist}
\item[\hskip \labelsep {\bfseries #1}\hskip \labelsep {\bfseries #2.}]}{\end{trivlist}}
\newcommand{\textfrac}[2]{\dfrac{\text{#1}}{\text{#2}}}

\begin{document}

\title{Real Analysis I: Assignment 2}

\author{Chris Hayduk}
\date{September 12, 2019}

\maketitle

\begin{problem}{1}
\end{problem}

Let the relation $R$ be $\leq$ on the set of real numbers $\mathbb{R}$. Now let $A = \mathbb{R} \cup c$ for some $c$ that cannot be compared to any $x \in \mathbb{R}$ and extend $R$ to this set $A$.\\

Since $c$ cannot be compared to any $x \in \mathbb{R}$, we have that $c \nleq x$ and $x \nleq c \; \forall x \in \mathbb{R}$. Thus, $c$ satisfies the definition of a minimal element in $A$, but does not satisfy the definition of the smallest element.\\

Now, let's fix a $y \in \mathbb{R}$ and assume it is a minimal element. Since $\mathbb{R}$ is closed under addition and we have $-1, y \in \mathbb{R}$, this yields $y + (-1) \in \mathbb{R}$. However, we know that $y + (-1) \leq y$. Thus, we have a contradiction and $\mathbb{R}$ has no minimal element under the relation $\leq$. Since the smallest element must also be minimal, $\mathbb{R}$ has no smallest element as well.\\

As a result, $A$ has a unique minimal element $c$ but has no smallest element.

\begin{problem}{2}
\end{problem}

Suppose that $\inf E < \sup E$ and that $E$ has only one element $b$. Then $\inf E \leq b \leq \sup E$.\\

From the definitions of infimum and supremum and the fact that $E$ contains only one element, we have that $\inf E + \epsilon > b$ and $\sup E - \epsilon < b \; \forall \epsilon > 0$. Since $\epsilon$ can be made arbitrarily close to 0, we have $\inf E = b = \sup E$.\\

This is a contradiction, and so if $\inf E < \sup E$, then $E$ must contain at least two elements.\\

Now suppose that $E$ has at least two elements.\\

Choose two elements $a, b \in E$. Since $\mathbb{R}$ is an ordered field and $E \subset \mathbb{R}$, we have that $a \leq b$ or $b \leq a$. Assume $a \leq b$ without loss of generality.\\

Since every element of a set is distinct, $a \neq b$ and we have $a < b$.\\

By definition of $\inf E$, we have that $\inf E \leq E \implies \inf E \leq a$. Similarly, we have $E \leq \sup E \implies b \leq \sup E$.\\

These two statements yield,
\begin{align*}
&\inf E \leq a < b \leq \sup E
\end{align*}

Thus, if $E$ has at least two elements, $\inf E < \sup E$.

\begin{problem}{3}
\end{problem}	

Suppose that a sequence $\left<x_n\right>$ has two distinct limits, $l_1$ and $l_2$.\\

Then $\forall \epsilon > 0, \exists N_1, N_2 \in \mathbb{N}$ such that $n \geq N_1 \implies | x_n - l_1 | < \frac{\epsilon}{2}$ and $n \geq N_2 \implies | x_n - l_2 | < \frac{\epsilon}{2}$.\\

Now, let $N = max\{N_1, N_2\}$ and fix $n \geq N$. Then, by the triangle inequality, we have
\begin{align*}
| l_1 - l_2 | \leq | x_n - l_1 | + | x_n - l_2 | < \frac{\epsilon}{2} + \frac{\epsilon}{2} = \epsilon
\end{align*}

Since $\epsilon$ can be made arbitrarily close to 0, the distance between $l_1$ and $l_2$ is arbitrarily close to 0 as well and hence, $l_1 = l_2$.\\

This is a contradiction to our supposition that $l_1$ and $l_2$ are distinct. As a result, a sequence can have at most one limit.

\begin{problem}{4}
\end{problem}

Suppose that a sequence $\left<x_n\right>$ is Cauchy.\\

That is, $\forall \epsilon > 0, \exists N \in \mathbb{N}$ such that whenever $n, m \geq N, | x_n - x_m | < \epsilon$.\\

Now take $\epsilon = 1$. Then we have (taking $n \geq N$),
\begin{align*}
& x_n = | (x_n - x_N) + x_N | \leq | x_n - x_N | + x_N \leq \epsilon + x_N = 1 + x_N
\end{align*}

for all $n \geq N$.\\

Let $M = 1 + max\{x_1, x_2, ..., x_N\}$. Then it is clear that M defines an upper bound for every term in $\left<x_n\right>$. Thus, $\left<x_n\right>$ is a bounded sequence and we can apply the Bolzano-Weierstrass theorem. This theorem states that each bounded sequence contains a convergence subsequence.\\

Define a subsequence $\left<x_{n_k}\right>$ such that the sequence converges to some limit $l$. Fix $\epsilon$ and define $N_1$ such that whenever $n_k \geq N_1, | x_{n_k} - l | < \frac{\epsilon}{2}$. Now, define an $N_2$ for the parent sequence $\left<x_n\right>$ such that whenever $n, m \geq N_2, | x_n - x_m | < \frac{\epsilon}{2}$.\\

Now take $N = max\{N_1, N_2\}$. It is clear then that we have the following,
\begin{align*}
| x_n - l | = | x_n - x_{n_k} + x_{n_k} - l | \leq | x_n - x_{n_k} | + | x_{n_k} - l | < \frac{\epsilon}{2} + \frac{\epsilon}{2} = \epsilon
\end{align*}

whenever $n, n_k \geq N$.\\

This satisfies the definition of a limit and so $\left<x_n\right> \to l$.

\end{document}