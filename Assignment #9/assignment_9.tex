\documentclass[12pt]{article}
 
\usepackage[margin=1in]{geometry}
\usepackage{amsmath,amsthm,amssymb, mathtools}
\usepackage[T1]{fontenc}
\usepackage{lmodern}
\usepackage{fixltx2e}
\usepackage[shortlabels]{enumitem}
\usepackage{mathrsfs}
 
\newcommand{\N}{\mathbb{N}}
\newcommand{\R}{\mathbb{R}}
\newcommand{\Z}{\mathbb{Z}}
\newcommand{\Q}{\mathbb{Q}}
 
\newenvironment{theorem}[2][Theorem]{\begin{trivlist}
\item[\hskip \labelsep {\bfseries #1}\hskip \labelsep {\bfseries #2.}]}{\end{trivlist}}
\newenvironment{lemma}[2][Lemma]{\begin{trivlist}
\item[\hskip \labelsep {\bfseries #1}\hskip \labelsep {\bfseries #2.}]}{\end{trivlist}}
\newenvironment{exercise}[2][Exercise]{\begin{trivlist}
\item[\hskip \labelsep {\bfseries #1}\hskip \labelsep {\bfseries #2.}]}{\end{trivlist}}
\newenvironment{problem}[2][Problem]{\begin{trivlist}
\item[\hskip \labelsep {\bfseries #1}\hskip \labelsep {\bfseries #2.}]}{\end{trivlist}}
\newenvironment{question}[2][Question]{\begin{trivlist}
\item[\hskip \labelsep {\bfseries #1}\hskip \labelsep {\bfseries #2.}]}{\end{trivlist}}
\newenvironment{corollary}[2][Corollary]{\begin{trivlist}
\item[\hskip \labelsep {\bfseries #1}\hskip \labelsep {\bfseries #2.}]}{\end{trivlist}}
\newcommand{\textfrac}[2]{\dfrac{\text{#1}}{\text{#2}}}

\begin{document}

\title{Real Analysis I: Assignment 9}

\author{Chris Hayduk}
\date{November 7, 2019}

\maketitle

\begin{problem}{1}
\end{problem}

Let $f$ be measurable and $B$ be a Borel set. Since $B$ is Borel, let $\langle E_i \rangle$ be a sequence of open sets such that $B = \cup E_i$.\\

Since $f$ is measurable, we know that the inverse image of each open set under $f$ must be measurable. Thus, we know that $f^{-1}(E_i)$ is measurable for every $i$. As a result, we have
\begin{align*}
f^{-1}(B) &= f^{-1}(\cup E_i)\\
&= \cup f^{-1}(E_i)
\end{align*}

Thus, $f^{-1}(B)$ is measurable.

\begin{problem}{2}
\end{problem}

Let $f$ be a measurable real-valued function and let $g$ be a continuous function defined on $(-\infty, \infty)$.\\

We know continuous functions are measurable, hence $g$ is also measurable.\\

Now fix $c \in \mathbb{R}$. We have,
\begin{align*}
\{x: (g \circ f)(x) > c\} &= (g \circ f)^{-1}[(c, \infty)]\\
&= f^{-1}[g^{-1}[(c, \infty)]]
\end{align*}

We know that the inverse image of an open set under a continuous function is open. Thus, $O = g^{-1}[(c, \infty)]$ is an open set. In addition, from the previous problem, we know that the inverse image of an open set under a measurable function is measurable. Hence,
\begin{align*}
f^{-1}[g^{-1}[(c, \infty)]] = f^{-1}(O)
\end{align*}

is measurable. As a result, $\{x: (g \circ f)(x) > c\}$, and thus $g \circ f$ satisfies the definition of a measurable function since $c$ was arbitrary.
\newpage
\begin{problem}{3}
\end{problem}

Let $E = \mathbb{R}$. If $f_n = \chi_{[n, \infty)}$, then $f_n(x) \to 0$. Now let $\delta = 1$ and $\epsilon = 1$.\\

Choose $A \subset E$ with $mA < 1$ and let $N \in \mathbb{Z}$. For every $x \geq N$ such that $x \not\in A$, we have that 
\begin{align*}
|f_N(x) - f(x)| = |f_N(x) - 0| = |f_N(x)| \geq 1
\end{align*}

Thus, there does not exist a measurable set $A \subset E$ with $mA < \delta$ such that for all $x \not\in A$ and all $n \geq N$, $|f_n(x) - f(x)| < \epsilon$ when $E$ has infinite measure.

\begin{problem}{4}
\end{problem}

Let $\langle f_n \rangle$ be a sequence of measurable functions that converges to a real-valued function $f$ a.e. on a measurable set $E$ with $mE < \infty$. Let $\eta > 0$ be given.
\end{document}