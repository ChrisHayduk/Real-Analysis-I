\documentclass[12pt]{article}
 
\usepackage[margin=1in]{geometry}
\usepackage{amsmath,amsthm,amssymb, mathtools}
\usepackage[T1]{fontenc}
\usepackage{lmodern}
\usepackage{fixltx2e}
\usepackage[shortlabels]{enumitem}
\usepackage{mathrsfs}

 
\newcommand{\N}{\mathbb{N}}
\newcommand{\R}{\mathbb{R}}
\newcommand{\Z}{\mathbb{Z}}
\newcommand{\Q}{\mathbb{Q}}
 
\newenvironment{theorem}[2][Theorem]{\begin{trivlist}
\item[\hskip \labelsep {\bfseries #1}\hskip \labelsep {\bfseries #2.}]}{\end{trivlist}}
\newenvironment{lemma}[2][Lemma]{\begin{trivlist}
\item[\hskip \labelsep {\bfseries #1}\hskip \labelsep {\bfseries #2.}]}{\end{trivlist}}
\newenvironment{exercise}[2][Exercise]{\begin{trivlist}
\item[\hskip \labelsep {\bfseries #1}\hskip \labelsep {\bfseries #2.}]}{\end{trivlist}}
\newenvironment{problem}[2][Problem]{\begin{trivlist}
\item[\hskip \labelsep {\bfseries #1}\hskip \labelsep {\bfseries #2.}]}{\end{trivlist}}
\newenvironment{question}[2][Question]{\begin{trivlist}
\item[\hskip \labelsep {\bfseries #1}\hskip \labelsep {\bfseries #2.}]}{\end{trivlist}}
\newenvironment{corollary}[2][Corollary]{\begin{trivlist}
\item[\hskip \labelsep {\bfseries #1}\hskip \labelsep {\bfseries #2.}]}{\end{trivlist}}
\newcommand{\textfrac}[2]{\dfrac{\text{#1}}{\text{#2}}}

\begin{document}

\title{Real Analysis I: Assignment 1}

\author{Chris Hayduk}
\date{\today}

\maketitle

\begin{problem}{1}
\end{problem}

Suppose $f: X \to Y$ is onto.\\

Let $B \subset Y$ be nonempty. Thus, $\exists b \in B$.\\

Suppose $f^{-1}(B) = \emptyset$. This implies that there is no $x \in X$ such that $f(x) = b$.\\

However, by the definition of onto, $\forall y \in Y$, $\exists x \in X$ such that $f(x) = y$. Since $b \in B \subset Y$ and $f$ is assumed to be onto, this definition applies.\\

Thus, we have a contradiction and $f^{-1}(B) \neq \emptyset$.

\begin{problem}{2}
\end{problem}

Let $\mathscr{A}$ be a collection of sets and assume properties (ii) and (iii) from the question.\\

Let $A, B \in \mathscr{A}$. By property (ii), $\tilde{A}, \tilde{B} \in \mathscr{A}$ as well.\\

Thus, by property (iii),
\begin{align*}
\tilde{A} \, \cap \, \tilde{B} &\in \mathscr{A}
\end{align*}

Then, by (ii) and DeMorgan's Laws, we have,
\begin{align*}
&\widetilde{\tilde{A} \cap \tilde{B}} \in \mathscr{A}\\
\implies &\tilde{\tilde{A}} \, \cup \, \tilde{\tilde{B}} \in \mathscr{A}\\
\implies &A \, \cup \, B \in \mathscr{A}
\end{align*}

Thus, by properties (ii) and (iii), whenever $A, B \in \mathscr{A}$, $A \cup B \in \mathscr{A}$ as well.

\newpage
\begin{problem}{3}
\end{problem}

\end{document}