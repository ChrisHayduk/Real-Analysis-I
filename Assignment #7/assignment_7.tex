\documentclass[12pt]{article}
 
\usepackage[margin=1in]{geometry}
\usepackage{amsmath,amsthm,amssymb, mathtools}
\usepackage[T1]{fontenc}
\usepackage{lmodern}
\usepackage{fixltx2e}
\usepackage[shortlabels]{enumitem}
\usepackage{mathrsfs}
 
\newcommand{\N}{\mathbb{N}}
\newcommand{\R}{\mathbb{R}}
\newcommand{\Z}{\mathbb{Z}}
\newcommand{\Q}{\mathbb{Q}}
 
\newenvironment{theorem}[2][Theorem]{\begin{trivlist}
\item[\hskip \labelsep {\bfseries #1}\hskip \labelsep {\bfseries #2.}]}{\end{trivlist}}
\newenvironment{lemma}[2][Lemma]{\begin{trivlist}
\item[\hskip \labelsep {\bfseries #1}\hskip \labelsep {\bfseries #2.}]}{\end{trivlist}}
\newenvironment{exercise}[2][Exercise]{\begin{trivlist}
\item[\hskip \labelsep {\bfseries #1}\hskip \labelsep {\bfseries #2.}]}{\end{trivlist}}
\newenvironment{problem}[2][Problem]{\begin{trivlist}
\item[\hskip \labelsep {\bfseries #1}\hskip \labelsep {\bfseries #2.}]}{\end{trivlist}}
\newenvironment{question}[2][Question]{\begin{trivlist}
\item[\hskip \labelsep {\bfseries #1}\hskip \labelsep {\bfseries #2.}]}{\end{trivlist}}
\newenvironment{corollary}[2][Corollary]{\begin{trivlist}
\item[\hskip \labelsep {\bfseries #1}\hskip \labelsep {\bfseries #2.}]}{\end{trivlist}}
\newcommand{\textfrac}[2]{\dfrac{\text{#1}}{\text{#2}}}

\begin{document}

\title{Real Analysis I: Assignment 7}

\author{Chris Hayduk}
\date{October 24, 2019}

\maketitle

\begin{problem}{1}
\end{problem}

Let $C = [0, 1] \sim [\cup O_n]$, where $O_n$ is defined as in the problem. Then $C = [0, 1] \cap \widetilde{[\cup O_n]}$. Since each $O_n$ is an open interval, and we know that a countable union of open sets is open, then $\cup O_n$ is open as well and, hence, $\widetilde{[\cup O_n]}$ is closed. Since $[0, 1]$ is closed, we then have that $C$ is closed.\\

In addition, by Theorem 12, we know that each closed set is measurable, so $C$ is measurable.\\

Now take $C_k = [0, 1] \sim [\bigcup_{n=1}^{k} O_n]$. From the construction of $C$, we know that $C_k$ is the disjoint union of $2^k$ closed intervals, each with length $\frac{1}{3^k}$.\\

Since $C = C_k \sim [\bigcup_{n=k+1}^{\infty} O_n]$, we have that $C \subset C_k$ for every $k \in \mathbb{N}$. Hence, $m(C) \leq m(C_k)$ for every $k$.\\

Thus, by the countable additivity of Lebesgue measure and the property that, for any interval $I, m(I) = \ell(I)$, we have
\begin{align*}
m(C) \leq m(C_k) &= \sum_{n=1}^{2^k} \frac{1}{3^k}\\
&= 2^k\left(\frac{1}{3^k}\right) = \left(\frac{2}{3}\right)^k
\end{align*}

Since $m(C) \leq \left(\frac{2}{3}\right)^k$ for any choice of $k$ and $\left(\frac{2}{3}\right)^k \to 0$ as $k \to \infty$, we have that $m(C) = 0$.

\begin{problem}{2}
\end{problem}

Fix $\epsilon > 0$ and assume there is a finite union $U$ of open intervals such that $m^*(U \Delta E) < \epsilon/3$. That is, $m^*(U \sim E) + m^*(E \sim U) < \epsilon/3$.\\

Moreover, there is an open set $V$ such that $E \sim U \subset V$ and $m^*V \leq m^*(E \sim U) + \epsilon/3$.\\

Hence, we have that $E \subset U \cup V = O$ and,
\begin{align*}
m^*(O \sim E) &= m^*((U \cup V) \sim E)\\
&= m^*((U \sim E) \cup (V \sim E))\\
&\leq m^*((U \sim E) \cup (E \sim U) \cup (V \sim (E \sim U)))\\
&\leq m^*(U \sim E) + m^*(E \sim U) + m^*(V \sim (E \sim U))\\
&< \epsilon/3 + \epsilon/3 + \epsilon/3 = \epsilon
\end{align*}

\begin{problem}{3}
\end{problem}

We have, by countable subadditivity, that,
\begin{align*}
m^*\left(A \cap \bigcup_{i=1}^{\infty} E_i\right) &= m^*\left(\bigcup_{i=1}^{\infty} A \cap E_i\right)\\
&\leq \sum_{i=1}^{\infty} m^*(A \cap E_i)
\end{align*}

In addition, we have for every $n \in \mathbb{N}$,
\begin{align*}
&\left(A \cap \bigcup_{i=1}^{\infty} E_i\right) \supset \left(\bigcup_{i=1}^{n} A \cap E_i\right)\\
\implies &m^*\left(A \cap \bigcup_{i=1}^{\infty} E_i\right) \geq m^*\left(\bigcup_{i=1}^{n} A \cap E_i\right)\\
\implies &m^*\left(A \cap \bigcup_{i=1}^{\infty} E_i\right) \geq \sum_{i=1}^{n} m^*(A \cap E_i)
\end{align*}

Since the left side of this inequality does not depend on the choice of $n$, we have that, 
\begin{align*}
m^*\left(A \cap \bigcup_{i=1}^{\infty} E_i\right) \geq \sum_{i=1}^{\infty} m^*(A \cap E_i)
\end{align*}

Thus,
\begin{align*}
m^*\left(A \cap \bigcup_{i=1}^{\infty} E_i\right) = \sum_{i=1}^{\infty} m^*(A \cap E_i)
\end{align*}

\begin{problem}{4}
\end{problem}

Suppose $E$ is measurable and $E \subset P$. Define $E_i = E \mathring{+} r_i$, where $r_i$ is defined as in the proof from our notes. Then $E_i \subset P_i$ for every $i$.\\

In addition, since $<P_i>$ is a disjoint sequence of measurable sets and $E_i \subset P_i$, we have that $<E_i>$ is a disjoint sequence of measurable sets as well. Moreover, $\cup E_i \subset \cup P_i \subset [0, 1)$.\\

From the above statements, we have that,
\begin{align*}
m(\cup E_i) = \sum mE_i \leq m([0, 1)) = 1
\end{align*}

Now, measure is modulo addition invariant, so $mE = mE_i$ for every $i$. Suppose $mE > 0$. Then,
\begin{align*}
\sum mE = \sum E_i = m(\cup E_i) \to \infty
\end{align*}

Thus, in order for $m(\cup E_i) < m([0, 1)) = 1$ to hold, $mE_i = mE = 0$.

\end{document}